\chapter{Macros}

There are two kinds of macros in ATS. One kind is C-like and the other one
is LISP-like, though they are much simpler as well as weaker than their
counterparts in C and LISP, respectively.

\section{C-like Macros}

We use some examples to illustrate certain typical uses of C-like macros in
ATS.

The following two declarations bind the identifiers $N_1$ and $N_2$ to {\em
the abstract syntax trees} (rather than strings) that represent $1024$ and
$N_1 + N_1$, respectively:
\begin{verbatim}
#define N1 1024; #define N2 N1 + N1
\end{verbatim}

Suppose we have the following value declaration appearing in the scope of
the above macro delarations:
\begin{verbatim}
val x = N1 * N2
\end{verbatim}

Then $N_1 * N_2$ first expands into $1024 * (N_1 + N_1)$, which further
expands into $1024 * (1024 + 1024)$. Note that if this example is done in
C, then $N_1 * N_2$ expands into $1024 * 1024 + 1024$, which is different
from what we have in ATS.  Also note that it makes no diffierence if we
reverse the order of the previous macro definitions:
\begin{verbatim}
#define N2 N1 + N1; #define N1 1024
\end{verbatim}
If we declare a marco as follows:
\begin{verbatim}
#define LOOP (LOOP + 1)
\end{verbatim}
then an infinite loop is entered (or more precisely, some macro expansion
depth is to be reached) when the identifier ${\it LOOP}$ is expanded.

\section{LISP-like Macros}
There are two forms of LISP-like macros in ATS: short form and long form.
These (untyped) macros are highly flexible and expressive, and they
{\em can} certainly be used in convoluted manners that should probably be
avoided in the first place. Some commonly used macro definitions can be
found in the following file:
\begin{center}
{\it \$ATSHOME/prelude/macrodef.sats}
\end{center}
In order to use LISP-like macros in ATS effectively, the programmer may
want to find some examples in LISP involving backquote-comma-notation.

\subsection{Macros in Long Form}
As a macro in short form can simply be considered a special kind of macro
in long form, we first give some explanantion on the latter.  A macro
definition in long form is introduced via the use of the keyword
{\it macrodef}. For instance, the following syntax introduces a macro name
${\it one}$ that refers to some code, that is, abstract syntax tree (AST)
representing the integer number $1$.
\begin{verbatim}
macrodef one = `(1)
\end{verbatim}

The special syntax \verb'`(...)', where no space is allowed between the
backquote symbol and the left parenthsis symbol, means to form an abstract
syntax tree representing what is written inside the parentheses.  This is
often referred to as backquote-notation. Intuitively, one may think that a
backquote-notation exerts an effect that {\it freezes} everything inside it.

Let us now define another macro as follows:
\begin{verbatim}
macrodef one_plus_one = `(1 + 1)
\end{verbatim}
The defined macro name ${\it one\_plus\_one}$ refers to some code (i.e.,
AST) representing $1 + 1$. At this point, it is important to stress that
the code representing $1 + 1$ is different from the code representing
<i>2</i>.  The macro name ${\it one\_plus\_one}$ can also be defined as
follows:
\begin{verbatim}
macrodef one_plus_one = `(,(one) + ,(one))
\end{verbatim}

The syntax \verb',(...)', where no space is allowed between the comma
symbol and the left parenthesis symbol, indicates the need to expand (or
evaluate) whatever is written inside the parentheses. This is often
referred to as comma-notation. A comma-notation is only allowed inside a
backquote-notation. Intuitively, a comma-notation cancels out the {\it
freezing} effect of the enclosing backquote-notation.

In addition to macro names, we can also define macro functions. For
instance, the following syntax introduces a macro function
${\it square\_mac}$:
\begin{verbatim}
macrodef square_mac (x) = `(,(x) * ,(x)) // [x] should refers to some code
\end{verbatim}

Here are some examples that make use of ${\it square\_mac}$:
\begin{verbatim}
fun square_fun (i: int): int = ,(square_mac `(i))
fun area_of_circle_fun (r: double): doubld = 3.1416 * ,(square_mac `(r))
\end{verbatim}

\subsection{Macros in Short Form}
The previous macro function ${\it square\_mac}$ can also be defined as
follows:
\begin{verbatim}
macdef square_mac (x) = ,(x) * ,(x) // [x] should refers to some code
\end{verbatim}

The keyword ${\it macdef}$ introduces a macro definition in short form.
The previous examples that make use of ${\it square\_mac}$ can now be
written as follows:
\begin{verbatim}
fun square_fun (i: int): int = square_mac (i)
fun area_of_circle_fun (r: double): doubld = 3.1416 * square_mac (r)
\end{verbatim}

In terms of syntax, a macro function in short form is just like an ordinary
function.  In general, if a unary macro function ${\it fmac}$ in short
form is defined as as follows:
\begin{verbatim}
macdef fmac (x) = fmac_body
\end{verbatim}
where ${\it fmac\_body}$ refers to some dynamic expression, then one may
essentially think that a macro definition in long form is defined as
follows:
\begin{verbatim}
macrodef fmac_long (x) = `(fmac_body) // please note the backquote
\end{verbatim}
and each occurrence of ${\it fmac}({\it arg})$ is automatically rewritten
into $,({\it fmac\_long}(`({\it arg})))$, where ${\it arg}$ refers to a
dynamic expression. Note that macro functions in short form with multiple
arguments are handled in precisely the same fashion.

The primary purpose for introducing macros in short form is to provide a
form of syntax that seems more accessible. While macros in long form can be
defined recursively (as is to be explained later), macros in short form
cannot.

\subsection{Recursive Macro Definitions}

%%% end of \chapter{Macros}


