\chapter{BATCH COMPILATION}\label{chapter:batch_compilation}

The command for compiling ATS programs into C code is {\it atsopt}.  After
C code is emitted by {\it atsopt}, it can then be compiled into machine
code by {\it gcc}. The command {\it atscc}, which essentially combines {\it
atsopt} and {\it gcc}, compiles ATS programs directly into machine code.
Both {\it atsopt} and {\it atscc} are implemented in ATS. If a
C compiler other than {\it gcc} is to be used, the command name for
this C compiler needs to be defined in the environment variable
{\it ATSCCOMP}.

\section{The Command {\it atsopt}}
The command {\it atsopt} compiles ATS programs into C code.  This command
is primarily used in scripting files such as those needed by the {\it make}
command.

\subsection{Compiling Static and Dynamic Files}
The following command line compiles a dynamic file {\it foo.dats} into C
code:
\begin{verbatim}
atsopt -d foo.dats
\end{verbatim}
and the output is directed to {\it stdout}.  The flag \verb`-d` may be
replaced with `--dynamic`.

Suppose it is desired to store the emitted C code into a file with the name
{\it foo\_dats.c}, the following command line can be issued:
\begin{verbatim}
atsopt -o foo_dats.c -d foo.dats
\end{verbatim}
The flag \verb`-o` may also be replaced with `--output`.
It should be emphasized that the part
\verb`-o foo_dats.c` must be put in front of \verb`-d foo.dats`.

Similarly, the following command line compiles a static file {\it foo.sats}
into C code:
\begin{verbatim}
atsopt -s foo.sats
\end{verbatim}
and the output is directed to {\it stdout}.  The flag \verb`-s` may be
replaced with `--static`. If the following command line is issued (and
executed successfully), then the C code emitted from compiling {\it
foo.sats} and {\it foo.dats} is to be stored in files {\it foo\_sats.c} and
{\it foo\_dats.c}, respectively.

\begin{verbatim}
atsopt -o foo_sats.c -s foo.sats -o foo_dats.c -d foo.dats
\end{verbatim}

\subsection{Typechecking Only}
If the following command line is issued, then the file {\it foo.dats} is
only typechecked:
\begin{verbatim}
atsopt -tc -d foo.dats
\end{verbatim}
The flag \verb`-tc` may be replaced with `--typecheck`.  Even if the file
{\it foo.dats} passes typechecking, no efffort is to be made to emit C code
from compiling {\it foo.dats}. If both {\it foo.dats} and {\it foo.sats}
need to be typechecked, it can be done as follows:
\begin{verbatim}
atsopt -tc -d foo.dats -s foo.sats
\end{verbatim}

\subsection{Generating HTML Files}
The flag \verb`--posmark_html` can be used to turn ATS programs into HTML
files for the purpose of viewing (through a browser). For instance, if the
following command line is issued, a file {\it foo\_dats.html} is generated:
\begin{verbatim}
atsopt --posmark_html -d foo.dats > foo_dats.html
\end{verbatim}

\subsection{Generating HTML Files for cross-referencing}
The flag \verb`--posmark_xref` can be used to turn ATS programs into HTML
files for the purpose of viewing and cross-referencing (through a
browser). For instance, if the following command line is issued, a file
{\it foo\_dats.html} is generated while various other files are created
in the directory {\it XREF/} for the purpose of cross-referencing:
\begin{verbatim}
atsopt --posmark_xref=XREF -d foo.dats > foo_dats.html
\end{verbatim}

\subsection{Generating Usage Information}
The following command line can be issued to generate some brief information
on the usage of {\it atsopt}:
\begin{verbatim}
atsopt -h
\end{verbatim}
The flag \verb`-h` may be replaced with `--help`.

\subsection{Generating Version Information}
The following command line can be issued to generate version information
on {\it atsopt}:
\begin{verbatim}
atsopt -v
\end{verbatim}
The flag \verb`-v` may be replaced with `--version`.

%%% end of \section{The command {\it atsopt}} %%%

\section{The Command {\it atscc}}

The command {\it atscc} combines {\it atsopt} and {\it gcc}, and it is
designed to be used in both command lines and scripting files.  Explanation
on special flags for {\it atscc} is given as follows. If a flag is not
special to {\it atscc}, it is passed to {\it gcc} directly by {\it atscc}.

\subsection{Generating Executables}
The following command line, if executed successfully, generates an
executable file {\it a.out}:
\begin{verbatim}
atscc foo.dats foo.sats
\end{verbatim}
Essentially, this command line is equivalent to the following one:
\begin{verbatim}
atsopt -o foo_dats.c -d foo.dats -o foo_sats.c -s foo.sats ; \
gcc -I $ATSHOME -I $ATSHOME/ccomp/runtime -L $ATSHOME/ccomp/lib foo_dats.c -lats
\end{verbatim}

If the generated executable needs to be given the name {\it foo}, then
the following command line can be issued:
\begin{verbatim}
atscc -o foo foo.dats foo.sats
\end{verbatim}
Of course, flags for {\it gcc} such as \verb`-O2`, \verb`-Wall`
and \verb`-fomit-frame-pointer` can be added freely as is done in
the following command line:
\begin{verbatim}
atscc -O2 -o foo -Wall -fomit-frame-pointer foo.dats foo.sats
\end{verbatim}

\subsection{Typechecking Only}
The flag \verb`-tc` or \verb`--typecheck` is used to indicate typechecking
only. For instance, the following command line:
\begin{verbatim}
atscc -tc foo.sats foo.dats
\end{verbatim}
is equivalent to the one below:
\begin{verbatim}
atsopt -tc --static foo.sats --dynamic foo.dats
\end{verbatim}

\subsection{Compilation Only}
The flag \verb`-cc` or \verb`--compile` is used to indicate compilation only.
For instance, the following command line:
\begin{verbatim}
atscc -cc foo.sats foo.dats
\end{verbatim}
is equivalent to the one below:
\begin{verbatim}
atsopt -o foo_sats.c --static foo.sats -o foo_dats.c --dynamic foo.dats
\end{verbatim}

\subsection{Binary Types}
The flags \verb`-m32` and \verb`-m64` can be passed to indicate the need
for generating binaries running on 32-bit and 64-bit machines, respectively.

\subsection{Garbage Collection}
By default, executables generated by {\it atscc} run without garbage
collection (GC). If executables need to be generated that run with GC being
turned on, the flag \verb`-D_ATS_GCATS` should be present.  For instance,
the follow command line generates such an executable named {\it foo}:
\begin{verbatim}
atscc -D_ATS_GCATS -o foo foo.dats foo.sats
\end{verbatim}
Note that the flag \verb`-D_ATS_GCATS` should only be used when {\it atscc}
is called to generate executables.

\subsection{Directories for File Search}
The use of \verb`-IATS` by {\it atscc} is analogous to \verb`-I` by {\it gcc}.
By default, {\it atscc} searches for files only in the directory {\ATSHOME}
and the current directory. If more directories need to be searched, it can
be added as follows:
\begin{verbatim}
atscc -IATS barpath1 -IATS barpath2 foo.sats foo.dats
\end{verbatim}
where {\it barpath1} and {\it barpath2} represent paths to some existing
directories.  The space following \verb`-IATS` is optional and it can be
erased if desired. Note that this command-line feature also applies to the
command {\it atsopt}.

\subsection{Setting Command-Line Flags}
The use of \verb`-DATS` by {\it atscc} is analogous to \verb`-D` by {\it gcc}.
For instance, the following command-line:
\begin{verbatim}
atscc -DATS FOO=123 -DATS BAR=xyz foo.sats foo.dats
\end{verbatim}
first sets {\it FOO} and {\it BAR} to strings "123" (instead of the integer
{\it 123}) and "xyz", respectively, and then compiles the files foo.sats
and foo.dats.  The space following \verb`-DATS` is optional and it can be
erased if desired.  Note that this command-line feature also applies to the
command {\it atsopt}.

%%% end of \chapter{BATCH COMPILATION} %%%
