\documentclass[pdf]{prosper}
\usepackage{amssymb}

\renewcommand{\slidetopmargin}{25mm}
\renewcommand{\slidebottommargin}{12mm}
\renewcommand{\slideleftmargin}{0mm}
\renewcommand{\sliderightmargin}{32mm}

%%%%%%
%%
%% Title: Views and Viewtypes in ATS
%% Author: Hongwei Xi, Boston University
%%
%% Date: December 5th, 2011
%% Time: 4-5PM
%%
\title{\huge\bf Views and Viewtypes in ATS}
\author{~\\~\\
{\large Hongwei Xi} \\~\\
{\large Boston University} \\~\\~\\~\\
\institution{Work partly funded by NSF}}
\slideCaption{\footnotesize{\em Views and Viewtypes in ATS}}
%%%%%%
%%
%% Abstract
%%
%% ATS is a statically typed general-purpose programming language.  The
%% signatory feature of ATS is a programming paradigm named
%% programming-with-theorem-proving in which code for (run-time) computation
%% and code for proof construction can be written in a syntactically
%% intertwined manner.  In ATS, there is direct support for both dependent
%% types and linear types. While the dependent types in ATS, which are of a
%% restricted form originated from the development of Dependent ML (DML), are
%% well-studied, the linear types in ATS are much less well-known.  In this
%% talk, I will give an introduction to a notion of linear types referred to
%% as viewtypes in ATS, which combine views (i.e., linear types for proofs)
%% and types (for run-time values). In addition, I will present several
%% concrete examples to illustrate certain practical uses of views and
%% viewtypes.
%%
%%%%%%
\begin{document}
%%%%%%
%%
%% Slide 1
%%
%%%%%%
\maketitle
%%%%%%
%%
%% Slide 2
%%
%%%%%%
\begin{slide}{\em ATS(1)}
\begin{itemize}
\item
ATS is a statically typed programming language that unifies implementation
with formal specification.
\item
The type system of ATS is rooted in the framework {\em Applied Type System},
which gives the language its name.
\item
Both dependent types (DML-style) and linear types are available in ATS.
\item
The current compiler of ATS (ATS/Anairiats) is written in ATS itself,
consisting of approximately 100K lines of code.
\end{itemize}
\end{slide}
%%%%%%
%%
%% Slide 3
%%
%%%%%%
\begin{slide}{\em ATS(2)}
\begin{itemize}
\item
The core of ATS is a call-by-value functional programming language, which
can also support lazy evaluation.
\item
ATS supports a programming paradigm referred to as {\em programming with
theorem-proving (PwTP)}, allowing code for run-time computation and
code for proof construction to be written in a syntactically intertwined
manner.
\item
Imperative programming in ATS makes essential use of PwTP.
\item
ATS and C share the same native data representation, and
programs in ATS are compiled directly into C code. To some extent, ATS
can be thought of as a front end for C.
\item
For more details, please visit:
\begin{center}
\texttt{http://www.ats-lang.org}
\end{center}
\end{itemize}
\end{slide}
%%%%%%
%%
%% Slide 4
%%
%%%%%%
\def\ATS{\mbox{$\cal A\kern-1.5pt T\kern-3pt S$}}
\begin{slide}{\em Applied Type System ($\ATS$)}
\begin{itemize}
\item
$\ATS$ is a framework developed to facilitate designing and
formalizing (advanced) type systems in support of practical
programming.
\item
The name {\em applied type system} refers to a type system formed in
$\ATS$, which consists of two components:
\begin{itemize}
\item
static component (statics), where types are formed and reasoned about.
\item
dynamic component (dynamics), where programs are constructed and evaluated.
\end{itemize}
\item
The key salient feature of $\ATS$ is that statics is completely separate
from dynamics. In particular, types cannot depend on programs.
\end{itemize}
\end{slide}
%%%%%%
%%
%% Slide 5
%%
%%%%%%
\begin{slide}{\em Examples and Non-Examples of ATS's}
\begin{itemize}

\item Examples:
\begin{itemize}
\item
The simply-typed $\lambda$-calculus
\item Dependent ML (DML)
\item
The 2nd-order polymorphic $\lambda$-calculus (System $F$)
\item
The higher-order polymorphic $\lambda$-calculus (System $F_{\omega}$)
\item
System $F_c$ (System $F$ extended with generalized datatypes)
\end{itemize}

\item Non-Examples:
\begin{itemize}
\item The dependent $\lambda$-calculus ($\lambda P$)
\item The calculus of constructions ($\lambda C$) and its variants
\end{itemize}

\end{itemize}
\end{slide}
%%%%%%
%%
%% Slide 6
%%
%%%%%%
\def\sc{{\it sc}}
\def\basesort{b}
\def\saddr{{\it addr}}
\def\sbool{{\it bool}}
\def\simp{\rightarrow}
\def\Simp{\Rightarrow}
\def\sint{{\it int}}
\def\snat{{\it nat}}
\def\sprop{{\it prop}}
\def\stype{{\it type}}
\begin{slide}{{\em Syntax for Statics}}
\begin{itemize}
\item
The statics is a simply typed language and a type in the statics is
referred to as a {\em sort}. We write $b$ for a base sort and assume the
existence of two special base sorts $\stype$ and $\sbool$.
\[\begin{array}{lrcl}
\mbox{sorts} & \sigma & ::= & \basesort\mid\sigma_1\simp\sigma_2 \\
\mbox{c-sorts} & \sigma_c & ::= & (\sigma_1,\ldots.\sigma_n)\Simp\sigma \\
\mbox{sta.cd  terms} & s & ::= & a \mid \sc(s_1,\ldots,s_n) \mid \lambda a:\sigma.s \mid s_1(s_2) \\
\mbox{sta. var. ctx.} & \Sigma & ::= & \emptyset \mid \Sigma, a:\sigma \\
\end{array}\]
\item
In practice, we also have base sorts $\sint$ and $\saddr$ for integers and
addresses (or locations), respectively.  Let us use $B$, $I$, $L$ and $T$
for static terms of sorts $\sbool$, $\sint$, $\saddr$ and $\stype$,
respectively.
\end{itemize}
\end{slide}
%%%%%%
%%
%% Slide 7
%%
%%%%%%
\def\Band{\land}
\def\Bimp{\supset}
\def\ttrue{{\it true}}
\def\ffalse{{\it false}}
\def\timp{\rightarrow}
\def\Timp{\Rightarrow}
\def\tbool{\mbox{\bf bool}}
\def\tint{\mbox{\bf int}}
\def\tleq{\leq}
\def\tpjg{\vdash}
\def\tunit{{\mathbf 1}}
\begin{slide}{\em Some Static Constants}
\[\begin{array}{rcll}
\\[-24pt]
\tunit & ~:~ & ()\Simp \stype \\[4pt]
\ttrue & ~:~ & ()\Simp \sbool \\[4pt]
\ffalse & ~:~ & ()\Simp \sbool \\[4pt]

\timp & ~:~ & (\stype,\stype) \Simp \stype \\[4pt]
\Bimp & ~:~ & (\sbool,\stype) \Simp \stype \\[4pt]

\Band & ~:~ & (\sbool,\stype) \Simp \stype \\[4pt]
\tleq & ~:~ & (\stype,\stype) \Simp \sbool & \mbox{(impredicative formulation)}\\[4pt]
\end{array}\]
Also, for each sort $\sigma$, we assume that the two static constructors
$\forall_\sigma$ and $\exists_\sigma$ are assigned the sc-sort
$(\sigma\timp\stype)\Simp\stype$.
\vfill
\end{slide}
%%%%%%
%%
%% Slide 8
%%
%%%%%%
\def\vB{\vec{B}}
\def\temd{\models}
\begin{slide}{\em Constraint Relation}
A constraint relation is of the following form:
\[\begin{array}{c}
\Sigma;\vB\temd B
\end{array}\]
where $\vB$ stands for a sequence of static boolean terms
(often referred to as assumptions). This relation holds if
one of the assumptions in $\vB$ is false or the conclusion $B$
is true.\\~\\
\end{slide}
%%%%%%
%%
%% Slide 9
%%
%%%%%%
\begin{slide}{\em A Sample Constraint}
\def\sz{{\it sz}}
The following constraint is generated when an
implementation of binary search on arrays is type-checked:
\[\begin{array}{rcl}
\Sigma;\vB & \temd & l+(h-l)/2+1\leq \sz \\
\end{array}\]
where
\[\begin{array}{rcl}
\Sigma & = & h:\sint, l:\sint, \sz:\sint \\
\vB & = & l\geq 0, \sz\geq 0, 0\leq h+1, h+1\leq \sz, 0\leq l, l\leq \sz, h\geq l \\
\end{array}\]
We may employ linear integer programming to solve such a constraint.\\~\\
\end{slide}
%%%%%%
%%
%% Slide 10
%%
%%%%%%
\begin{slide}{\em Some (Unfamiliar) Forms of Types}
\begin{itemize}
\item Asserting type: $B\Band T$
\item Guarded type: $B\Bimp T$
\end{itemize}
Here is an example involving both guarded and asserting types:
\[\begin{array}{l}
\forall a:\sint.~a \geq 0\Bimp (\tint (a) \timp \exists a':\sint.~(a'<0)\Band \tint(a')) \\
\end{array}\]
This type can be assigned to a function from nonnegative integers to
negative integers. As a probably more interesting example, the usual
run-time assertion function can be given the following type:
\[\begin{array}{l}
\forall a:\sbool.~\tbool(a)\timp(a=\ttrue)\Band\tunit \\
\end{array}\]
\vfill
\end{slide}
%%%%%%
%%
%% Slide 11
%%
%%%%%%
\def\dc{{\it dc}}
\def\dcc{{\it dcc}}
\def\dcf{{\it dcf}}
\def\iforall#1{\forall^{+}(#1)}
\def\eforall#1{\forall^{-}(#1)}
\def\iguard#1{\supset^{+}\kern-2pt(#1)}
\def\eguard#1{\supset^{-}\kern-2pt(#1)}
\def\lam#1#2{\mbox{\bf lam}\;#1.#2}
\def\app#1#2{\mbox{\bf app}(#1,#2)}
\def\letin#1#2{\mbox{\bf let}\;#1\;{\bf in}\;#2}
\begin{slide}{\em Syntax for Dynamics}
\[\begin{array}{lrcl}
\mbox{dyn. terms} & d & ::= & x \mid \dc(d_1,\ldots,d_n) \mid \\
& & & \lam{x}{d} \mid \app{d_1}{d_2} \mid \\
& & & \iguard{v} \mid \;\eguard{d} \mid \\
& & & \iforall{v} \mid \eforall{d} \mid \\
& & & \Band(d) \mid \letin{\Band(x)=d_1}{d_2} \mid \\
& & & \exists(d) \mid \letin{\exists(x)=d_1}{d_2} \\
\mbox{values} & v & ::= & x \mid \dcc(v_1,\ldots,v_n) \mid \lam{x}{d} \mid \\
& & & \iguard{v} \mid \iforall{v} \mid \Band(v) \mid \exists(v) \\
\mbox{dyn. var. ctx.} & \Delta & ::= & \emptyset \mid \Delta, x:s \\
\end{array}\]
\end{slide}
%%%%%%
%%
%% Slide 12
%%
%%%%%%
\begin{slide}{\em Typing Judgment}
A typing judgment is of the following form:
\[\begin{array}{c}
\Sigma;\vB;\Delta\tpjg d:T
\end{array}\]
where
\[\begin{array}{rcl}
\Sigma & : & \mbox{static variable context} \\
\vB & : & \mbox{assumption set} \\
\Delta & : & \mbox{dynamic variable context} \\
d & : & \mbox{dynamic term} \\
T & : & \mbox{type} \\
\end{array}\]
\end{slide}
%%%%%%
%%
%% Slide 13
%%
%%%%%%
\begin{slide}{\em A Datatype Declaration in ATS}
{\blue\begin{verbatim}
datatype list (a:type, int) =
  | nil (a, 0) of ()
  | {n:int | n >= 0} cons (a, n+1) of (a, list (a, n))
\end{verbatim}
}~\\~\\
The concrete syntax means the following:
\[\begin{array}{rcl}
nil & : & \forall a:\stype.~~() \Timp list (a, 0) \\
cons & : & \forall a:\stype.\forall n:\sint. \\
     &   & \kern12pt n\geq 0 \Bimp ((a, list (a, n)) \Timp list (a, n+1)) \\
\end{array}\]
\vfill
\end{slide}
%%%%%%
%%
%% Slide 14
%%
%%%%%%
\begin{slide}{\em Programming with Theorem-Proving}
\begin{itemize}
\item
We introduce a new sort $\sprop$ into the statics and use $P$ for static
terms of sort $\sprop$, which are often referred to as props.
\item
A prop is like a type, which is intended to be assigned to special dynamic
terms that we refer to as proof terms.
\item
A proof term is required to be pure and total, and it is to be erased
before program execution. In particular, we do not extract programs out of
proofs. Consequently, we can and do construct classical proofs.
\end{itemize}
\end{slide}
%%%%%%
%%
%% Slide 15
%%
%%%%%%
\begin{slide}{\em A Dataprop Declaration in ATS}
{\blue\begin{verbatim}
datatype FIB (int, int) =
  | FIB0 (0, 0) of () // fib(0) = 0
  | FIB1 (1, 1) of () // fib(1) = 1
  | {n:nat}{r0,r1:nat} // fib(n+2) = fib(n)+fib(n+1)
    FIB2 (n+2, r0+r1) of (FIB (n, r0), FIB (n+1, r1))
\end{verbatim}
}~\\~\\
The following function {\it fib} computes Fibonacci numbers:\\~\\
{\blue\begin{verbatim}
fun fib {n:nat}
  (n: int (n)): [r:nat] (FIB (n, r) | int (r))
// end of [fib]
\end{verbatim}
}
\vfill
\end{slide}
%%%%%%
%%
%% Slide 16
%%
%%%%%%
\begin{slide}{\em A Direct Implementation of {\it fib}}
{\blue\begin{verbatim}
implement fib (n) =
  case+ n of
  | 0 => (FIB0 () |  0)
  | 1 => (FIB1 () |  1)
  | _ =>> let
      val (pf0 | r0) = fib (n-2)
      val (pf1 | r1) = fib (n-1)
    in
      (FIB2 (pf0, pf1) | r0 + r1)
    end // end of [_]
// end of [fib]
\end{verbatim}
}
\end{slide}
%%%%%%
%%
%% Slide 17
%%
%%%%%%
\begin{slide}{\em Views}
A view is a linear prop.
\begin{itemize}
\item
Given a type $T$ and an address $L$, $T@L$ is a primitive view
meaning that a value of the type $T$ is stored at the location $L$.
\item
Given two views $V_1$ and $V_2$, we use $V_1\otimes V_2$ for a
view that joins $V_1$ and $V_2$ together.
\item
We also provide a means for forming recursive views. 
\end{itemize}
\end{slide}
%%%%%%
%%
%% Slide 18
%%
%%%%%%
\def\tptr{\mbox{\bf ptr}}
\def\fptrget{{\it ptrget}}
\def\fptrset{{\it ptrset}}
\begin{slide}{\em Some Built-in Functions}
{\blue\begin{verbatim}
fun{a:type}
ptrget {l:addr}
  (pf: a@l | p: ptr l): (a@l | a)
// end of [ptrget]

fun{a:type}
ptrset {l:addr}
  (pf: (a?)@l | p: ptr l, x: a): (a@l | void)
// end of [ptrset]
\end{verbatim}
}
\[\begin{array}{rcl}
\fptrget &~:~& \forall a:\stype.\forall l:\saddr.~~(a@l \mid \tptr(l)) \timp (a@l \mid a) \\
\fptrset &~:~& \forall a:\stype.\forall l:\saddr.~~(a?@l \mid \tptr(l), a)\timp (a@l \mid \tunit) \\
\end{array}\]
\end{slide}
%%%%%%
%%
%% Slide 19
%%
%%%%%%
\begin{slide}{\em Example: {\it Swap}}
{\blue\begin{verbatim}
fun{a:type}
swap {l1,l2:addr} (
  pf1: a@l1, pf2: a@l2 | p1: ptr l1, p2: ptr l2
) : (a@l1, a@l2 | void) = let
  val (pf1 | x1) = ptrget<a> (pf1 | p1)
  val (pf2 | x2) = ptrget<a> (pf2 | p2)
  val (pf1 | ()) = ptrset<a> (pf1 | p1, x2)
  val (pf2 | ()) = ptrset<a> (pf2 | p2, x1)
in
  (pf1, pf2 | ())
end // end of [swap]
\end{verbatim}
}
\end{slide}
%%%%%%
%%
%% Slide 20
%%
%%%%%%
\begin{slide}{\em Syntactical Convenience}
{\blue\begin{verbatim}
fun{a:type}
ptrget {l:addr} (pf: !a@l | p: ptr l): a

fun{a:type}
ptrset {l:addr}
  (pf: !(a?)@l >> a@l | p: ptr l, x: a): void
// end of [ptrset]
\end{verbatim}
}
\end{slide}
%%%%%%
%%
%% Slide 21
%%
%%%%%%
\begin{slide}{\em Another Implementation of {\it Swap}}
{\blue\begin{verbatim}
fun{a:type}
swap {l1,l2:addr} (
  pf1: !a@l1, pf2: !a@l2
| p1: ptr l1, p2: ptr l2
) : void = let
  val tmp = !p1 in !p1 := !p2; !p2 := tmp
end // end of [swap]
\end{verbatim}
}
\end{slide}
%%%%%%
%%
%% Slide 22
%%
%%%%%%
\def\varrayView{{\it array\_v}}
\def\cArrayNil{{\it array\_v\_nil}}
\def\cArrayCons{{\it array\_v\_cons}}
\begin{slide}{\em A Dataview Declaration in ATS}
{\blue\begin{verbatim}
dataview array_v (a:type, int, addr) =
  | {l:addr}
    array_v_nil (a,0,l)
  | {n:nat}{l:addr}
    array_v_cons (a,n+1,l) of (a@l, array_v (a,n,l+1))
\end{verbatim}}~\\~\\
The concrete syntax means the following:
\[\begin{array}{rcl}
\cArrayNil  &\kern-6pt:\kern-6pt& \forall a:\stype.\forall l:addr.~~\varrayView(a,0,l) \\
\cArrayCons &\kern-6pt:\kern-6pt& \forall a:\stype.\forall n:\snat.\forall l:addr. \\
            &\kern-6pt:\kern-6pt& \kern12pt (a@l, \varrayView(a,n,l+1))\timp\varrayView(a,n+1,l) \\
\end{array}\]
\end{slide}
%%%%%%
%%
%% Slide 23
%%
%%%%%%
\begin{slide}{\em Proof Functions for View Change}
{\blue\begin{verbatim}
prfun split
  {a:type}
  {n:int}{i:nat | i <= n}
  {l:addr} (
  pf: array_v (a, n, l)
) : (array_v (a, i, l), array_v (a, n-i, l+i))

prfun unsplit
  {a:type}
  {n1,n2:nat}{l:addr} (
  pf1: array_v (a, n1, l), pf2: array_v (a, n2, l+n1)
) : array_v (a, n1+n2, l)
\end{verbatim}
}
\end{slide}
%%%%%%
%%
%% Slide 24
%%
%%%%%%
\begin{slide}{\em Another Proof Function for View Change}
{\blue\begin{verbatim}
prfun takeout
  {a:type}
  {n:int}{i:nat | i < n}
  {l:addr} (
  pf: array_v (a, n, l)
) : (a @ l+i, a @ l+i -<lin> array_v (a, n, l))
\end{verbatim}
}
\end{slide}
%%%%%%
%%
%% Slide 25
%%
%%%%%%
\begin{slide}{\em Array Subscripting}
{\blue\begin{verbatim}
fun{a:type}
arrsub
  {n:int}{i:nat | i < n}
  {l:addr} (
  pf: !array_v (a, n, l) | p: ptr l, i: int i
) : a = x where {
  prval (pfat, fpf) = takeout {a}{n}{i} (pf)
  val x = ptrget<a> (pfat | p+i)
  prval () = pf := fpf (pfat)
} // end of [arrsub]
\end{verbatim}
}
\end{slide}
%%%%%%
%%
%% Slide 26
%%
%%%%%%
\def\VT{{\it VT}}
\def\sviewtype{{\it viewtype}}
\begin{slide}{\em Viewtypes}
\begin{itemize}
\item
A linear type in ATS is given the name viewtype, which is chosen to
indicate that a linear type consists of two parts: one part for views and
the other for types. We use $\sviewtype$ as the sort for viewtypes.
\item Given a view V and a type T, the
tuple (V | T) is a viewtype, where the bar symbol (|) is a separator (just
like a comma) to separate views from types.
\item
What seems a bit surprising is the opposite: For each viewtype VT, we may
assume the existence of a view V and a type T such that VT is equivalent to
(V | T). Formally, this T can be referred as VT?! in ATS. This somewhat
unexpected interpretation of linear types stresses that the linearity of a
viewtype comes entirely from the view part residing within it.
\end{itemize}
\vfill
\end{slide}
%%%%%%
%%
%% Slide 27
%%
%%%%%%
\begin{slide}{\em Generic Types for {\it ptrget} and {\it ptrset}}
{\blue\begin{verbatim}
fun{a:viewtype}
ptrget {l:addr}
  (pf: !a@l>> (a?!)@l | p: ptr l): a
// end of [ptrget]

fun{a:viewtype}
ptrset {l:addr}
  (pf: !(a?)@l >> a@l | p: ptr l, x: a): void
// end of [ptrset]
\end{verbatim}
}
\end{slide}
%%%%%%
%%
%% Slide 28
%%
%%%%%%
\begin{slide}{\em Example: Viewtype for Linear Closures}
Given two types A and B, a pointer to some address L where a closure
function is stored that takes a value of the type A to return a value of
the type B can be given the viewtype cloptr(A, B, L):\\~\\
{\blue\begin{verbatim}
viewtypedef cloptr
  (a:t@ype, b:t@ype, l:addr) =
  [env:t@ype] (((&env, a) -> b, env) @ l | ptr l)
// end of [cloptr]
\end{verbatim}
}~\\
In the function type \verb`(&env, a) -> b`, the symbol \verb`&` indicates
that the corresponding function argument is passed by reference, that is,
the argument is required to be a left-value and what is actually passed is
the address of the left-value.
\end{slide}
%%%%%%
%%
%% Slide 29
%%
%%%%%%
\def\vtList{{\it list\_vt}}
\def\cListvtNil{{\it list\_vt\_nil}}
\def\cListvtCons{{\it list\_vt\_cons}}
\begin{slide}{\em A Dataviewtype Declaration in ATS}
{\blue\begin{verbatim}
dataviewtype
list_vt (a:type, int) =
  | {n:nat}
    list_vt_cons (a,n+1) of (a, list_vt (a,n))
  | list_vt_nil (a, 0)
\end{verbatim}
}~\\~\\
The concrete syntax means the following:
\[\begin{array}{rcl}
\cListvtNil  &:& \forall a:\stype.~~\vtList(a,0) \\
\cListvtCons &:& \forall a:\stype.\forall n:\snat.~~\\
             & &\kern12pt(a, \vtList(a,n))\timp\vtList(a,n+1) \\
\end{array}\]
\end{slide}
%%%%%%
%%
%% Slide 30
%%
%%%%%%
\begin{slide}{\em Example: Reverse}
{\blue\begin{verbatim}
fun{a:type}
reverse {n:nat} (
  xs: list_vt (a, n)
) : list_vt (a, n) = revapp (xs, list_vt_nil)
\end{verbatim}
}
\end{slide}
%%%%%%
%%
%% Slide 31
%%
%%%%%%
\begin{slide}{\em Example: Reverse-Append}
{\blue\begin{verbatim}
fun{a:type}
revapp {m,n:nat} (
  xs: list_vt (a, m), ys: list_vt (a, n)
) : list_vt (a, m+n) =
  case+ xs of
  | list_vt_cons
      (_, !ptl) => let
      val tl = !ptl; val () = !ptl := ys
      prval () = fold@ (xs)
    in
      revapp (tl, xs)
    end
  | ~list_vt_nil () => ys
// end of [revapp]
\end{verbatim}
}
\end{slide}
%%%%%%
%%
%% Slide 32
%%
%%%%%%
\begin{slide}{\em Abstract Views}
{\blue\begin{verbatim}
absview free_v (n:int, l:addr)

fun free {n:nat}{l:addr} (
  pfgc: free_v (n, l), pfat: bytes(n) @ l | p: ptr l
) : void // end of [free]

dataview malloc_v (n:int, addr) =
  | {l:agz} malloc_v_succ (n, l) of
      (free_v (n, l), bytes(n) @ l | ptr l)
  | malloc_v_fail (n, null) of ()

fun malloc {n:nat}
  (n: size_t (n)): [l:addr] (malloc_v (n, l) | ptr l)
// end of [malloc]

\end{verbatim}
}
\end{slide}
%%%%%%
%%
%% Slide 33
%%
%%%%%%
\begin{slide}{\em Abstract Viewtypes (1)}
{\blue\begin{verbatim}
absviewtype queue (a:viewtype, n:int)

fun{a:viewtype} queue_new (): queue (a, 0)
fun{a:viewtype} queue_free (obj: queue (a, 0)): void

fun{a:viewtype}
queue_enque {n:nat} (
  obj: !queue (a, n) >> queue (a, n+1), x: a
) : void // end of [queue_enque]

fun{a:viewtype}
queue_deque {n:pos}
  (obj: !queue (a, n) >> queue (a, n-1)): a
\end{verbatim}
}
\end{slide}
%%%%%%
%%
%% Slide 34
%%
%%%%%%
\begin{slide}{\em Abstract Viewtypes (2)}
{\blue\begin{verbatim}
absviewtype mylock (v:view)

fun mylock_create
  {v:view} (pf: v | (*none*)): mylock(v)

fun mylock_destroy
  {v:view} (lock: mylock v): (v | void)

fun mylock_acquire
  {v:view} (lock: !mylock v): (v | void)

fun mylock_release
  {v:view} (pf: v | lock: !mylock v): void
\end{verbatim}
}
\end{slide}
%%%%%%
%%
%% Slide 35
%%
%%%%%%
\begin{slide}{\em Conclusion}
\begin{itemize}
\item
ATS is a statically type programming language that unifies implementation
with (a form of) formal specification.
\item
The signatory feature of ATS is a programming paradigm referred to as
programming-with-theorem-proving (PwTP) in which code for (run-time)
computation and code for proof construction can be written in a
syntactically intertwined manner.
\item
Views are linear props and viewtypes combine views with types.  In
particular, the linearity of a viewtype comes entirely from its view part.
Imperative programming in ATS is built on top of views and viewtypes
(together with the support of PwTP).
\end{itemize}
\end{slide}
%%%%%%
%%
%% Slide 36
%%
%%%%%%
\begin{slide}{\em Future Direction}
\begin{itemize}
\item Applying ATS to systems programming. We need more case studies.
\item
Reducing the amount of ``administrative code'' needed for proof
manipulation. We want to investigate more effective approaches to proof
management in order to better support PwTP.
\item
Identifying the potential of PwTP in areas other than systems programming
(e.g., security).
\end{itemize}
\end{slide}
%%%%%%
%%
%% Slide 37
%%
%%%%%%
\begin{slide}{\em The End}
\vspace{72pt}
\begin{center}
{\huge\bf Thank you!}
\end{center}
\end{slide}
\end{document}
%%%%%% end of [ViewsViewtypes.tex] %%%%%%
