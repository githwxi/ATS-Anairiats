\documentclass[pdf]{prosper}
\usepackage{amssymb}

\renewcommand{\slidetopmargin}{25mm}
\renewcommand{\slidebottommargin}{12mm}
\renewcommand{\slideleftmargin}{0mm}
\renewcommand{\sliderightmargin}{32mm}

\def\sc{{\it sc}}
\def\basesort{b}
\def\saddr{{\it addr}}
\def\sbool{{\it bool}}
\def\simp{\rightarrow}
\def\Simp{\Rightarrow}
\def\sint{{\it int}}
\def\snat{{\it nat}}
\def\sprop{{\it prop}}
\def\stype{{\it type}}

\def\tint{\mbox{\bf int}}
\def\tptr{\mbox{\bf ptr}}
\def\tbool{\mbox{\bf bool}}
\def\tvoid{\mbox{\bf void}}
\def\tlist{\mbox{\bf list}}
\def\tllist{\mbox{\bf list\_vt}}
\def\tarray{\mbox{\bf array}}

\def\Band{\land}
\def\Bimp{\supset}
\def\timp{\rightarrow}

%%%%%%
%%
%% NASAC@NanHang
%%
%% Title: ATS as a functional programming front-end for C
%% Author: Hongwei Xi, Boston University
%%
%% Date: October 21, 2012
%%
\title{\huge\bf ATS as a Functional Programming Front-end for C}
\author{~\\~\\
{\large Hongwei Xi} \\~\\
{\large Boston University} \\~\\~\\~\\
\institution{Work partly funded by NSF}}
\slideCaption{\footnotesize{\em ATS as a functional programming front-end for C}}
%%%%%%
%%
%% Abstract
%%
%% Typical system design is marked by tradeoffs between factors such as
%% efficiency and reliability, features and security, etc. As of now, most
%% embedded systems are written in unsafe programming languages that are
%% either untyped (e.g., assembly) or weakly typed (e.g., C).  Evidently, a
%% question of great interest is why people (in general) do not implement
%% these systems in existing safe languages such as Java (object-oriented) and
%% ML (functional). Among a variety of reasons, a particularly important one
%% is that low-level features such as explicit use of pointers and pointer
%% arithmetic are disallowed in such safe languages (in order to guarantee
%% safety). However, the very need to achieve optimal hardware usage often
%% demands the availability of such features. In addition, the requirement for
%% run-time garbage collection (GC) by safe languages can often become a great
%% or even insurmountable obstacle to handling real-time events (that have
%% hard deadlines).
%%
%% ATS is a full-fledged functional programming language with a type system
%% rooted in the framework Applied Type System. ATS can run without GC and it
%% supports low-level features such as pointers and pointer arithmetic. When
%% such features are used, the programmer is required to construct proofs
%% within ATS to justify the safety of their use. This is a programming
%% paradigm we refer to as programming with theorem-proving. In this talk, I
%% will primarily argue that ATS as a functional programming front-end to C
%% can be readily employed in the construction of safe and reliable embedded
%% systems. In particular, I will demonstrate some highly effective uses of
%% dependent types and linear types in tracking the allocation/deallocation
%% and manipulation of computing resources and thus preventing their potential
%% misuses.
%%
%%%%%%
\begin{document}
%%%%%%
%%
%% Title Slide
%%
%%%%%%
\maketitle
%%%%%%
%%
%% Slide 1.1
%%
%%%%%%
\begin{slide}{\em Why functional programming?}
\begin{itemize}
\item
It can be readily observed that functional programming (FP) style is often
closer than imperative programming style to high-level specifications that
need to be implemented.

\begin{itemize}
\item FP is more ``declarative'': it is more about what-to-do than how-to-do-it.
\end{itemize}

\item
Common Misconceptions of FP:
\begin{itemize}
\item FP is exotic and difficult to learn.
\item FP is inherently inefficient time-wise.
\item FP is inherently inefficient memory-wise; it requires the support of run-time garbage collection (GC).
\end{itemize}

\end{itemize}
\end{slide}
%%%%%%
%%
%% Slide 1.2
%%
%%%%%%
\begin{slide}{\em Functional programming in ATS}
\begin{itemize}
\item ATS is a typed language of functional programming core.
\item FP programs in ATS can run with and without the support of run-time GC.
\item
The efficiency of programs written in ATS can rival their counterparts in
C both time-wise and memory-wise. See the following benchmarking site
run by a third-party:
\begin{center}
\texttt{\bf http://shootout.alioth.debian.org/}
\end{center}
\end{itemize}
\end{slide}
%%%%%%
%%
%% Slide 2.1
%%
%%%%%%
\begin{slide}{\em Why types?}
\begin{itemize}

\item
Types are essentially a specification language for writing specifications
against which implementations can be checked {\em mechanically} at
compile-time.

\item
The tool for supporting this kind of static checking is called
{\em typechecker}.

\item
Common Misconceptions of Types:
\begin{itemize}
\item Types are exotic and difficult to learn.
\item Types are too simple for capturing ``interesting'' program properties.
\end{itemize}

\end{itemize}
\end{slide}
%%%%%%
%%
%% Slide 2.2
%%
%%%%%%
\begin{slide}{\em Types in ATS}
\begin{itemize}
\item Types in ATS are rich and expressive.
\item In particularly, there are dependent types and linear types in ATS.
\begin{itemize}
\item Dependent types, which are rooted in the first-order
predicate logic, bring precision into specification.
\item Linear types, which are (casually) rooted in a fragment of the linear
logic, allow direct modeling of resource allocation/deallocation and manipulation.
\end{itemize}
\end{itemize}
\end{slide}
%%%%%%
%%
%% Slide 2.3
%%
%%%%%%
\begin{slide}{\em Examples of Types in ATS (1)}

\begin{itemize}
\item Primitive singleton types for enhancing precision:
\begin{itemize}
\item Given a integer n, $\tint(n)$ is the type for the only integer whose
  value equals n.
\item Given a boolean b, $\tbool(b)$ is the type for the only boolean whose
  value equals b.
\item Given a memory location l, $\tptr(l)$ is the type for the only pointer to the
  location l.
\end{itemize}

\item Precise dependent types for primitive operations:
$$\begin{array}{rcl}
{+} & : & \forall i:int.\forall j:int.(\tint(i), \tint(j))\timp \tint(i+j) \\
{-} & : & \forall i:int.\forall j:int.(\tint(i), \tint(j))\timp \tint(i-j) \\
{*} & : & \forall i:int.\forall j:int.(\tint(i), \tint(j))\timp \tint(i*j) \\
\ldots & : & \ldots \\
\end{array}$$

\end{itemize}
\end{slide}
%%%%%%
%%
%% Slide 2.4
%%
%%%%%%
\begin{slide}{\em Examples of Types in ATS (2)}
\begin{itemize}
\item

The following type is for a function that takes an integer and returns
a non-negative integer:

$$\begin{array}{l}
\forall i:\sint.~\tint(i) \timp \exists j:\sint.~(j \geq 0)\Band\tint(j) \\
\end{array}$$

The following is a representation of the above type in the concrete syntax
of ATS:

$$\begin{array}{l}
\texttt{\{i:int\} int(i) -> [j:int | j >= 0] int(j)} \\
\end{array}$$

For instance, this type can be assigned to the absolute-value function
or the square function on integers.

\end{itemize}
\end{slide}
%%%%%%
%%
%% Slide 2.5
%%
%%%%%%
\begin{slide}{\em Examples of Types in ATS (3)}
\begin{itemize}
\item

The following type is for a function that takes an array with a valid index
(i.e., a natural number less the size of the array) and returns an element
in the array:
$$\begin{array}{l}
\forall a:\stype.\forall n:\sint.\forall i:\snat.~\\
\kern18pt(i<n)\Bimp(\tarray(a, n), \tint(i)) \timp a \\
\end{array}$$
The following is a representation of the above type in the concrete syntax
of ATS:
$$\begin{array}{l}
\texttt{\{a:type\}\{n:int\}}\\
\kern12pt\texttt{\{i:nat | i < n\}(array(a, n), int(i)) -> a}\\
\end{array}$$
Clearly, this type can be assigned to the array-subscripting
function.

\end{itemize}
\end{slide}
%%%%%%
%%
%% Slide 2.6
%%
%%%%%%
\begin{slide}{\em Examples of Types in ATS (4)}
\begin{itemize}
\item

The following type is for a function that takes two lists of length $m$ and
$n$, respectively, and returns a list of length $m+n$:

$$\begin{array}{l}
\forall a:\stype.\forall m:\snat.\forall n:\snat.~\\
\kern18pt(\tlist(a, m), \tlist(a, n)) \timp \tlist(a, m+n)\\
\end{array}$$

For instance, this type can be assigned to the function for concatenating
two given lists in functional style.

\end{itemize}
\end{slide}
%%%%%%
%%
%% Slide 2.7
%%
%%%%%%
\begin{slide}{\em Examples of Types in ATS (5)}
\begin{itemize}
\item

The following type is for a function that takes two linear lists of length
$m$ and $n$, respectively, and returns a linear list of length $m+n$:

$$\begin{array}{l}
\forall a:\stype.\forall m:\snat.\forall n:\snat.~\\
\kern18pt(\tllist(a, m), \tllist(a, n)) \timp \tllist(a, m+n)\\
\end{array}$$

Note that the two arguments of this function are no longer available
after a call to it returns.

\end{itemize}
\end{slide}
%%%%%%
%%
%% Slide 2.8
%%
%%%%%%
\begin{slide}{\em Examples of Types in ATS (6)}
\begin{itemize}
\item

There is a famous saying in English: {\em one cannot have the cake and
eat it!} This can be formalized as follows:
$$\begin{array}{rcl}
\texttt{eat} & : & \texttt{(cake) -> void} \\
\texttt{have\_and\_eat} & : & \texttt{(!cake) -> void} \\
\end{array}$$

The symbol $!$ indicates that the argument of the function
\texttt{have\_and\_eat} is still available after a call to it returns.

\end{itemize}
\end{slide}
%%%%%%
%%
%% Slide 3.1
%%
%%%%%%
\begin{slide}{\em ATS (1)}
\begin{itemize}

\item
ATS is a statically typed programming language that unifies implementation
with formal specification.

\item
The type system of ATS is rooted in the framework {\em Applied Type System},
which gives the language its name.

\item
Both dependent types (based on first-order logic) and linear types are
available in ATS.

\item
The currently released compiler of ATS (ATS/Anairiats) is written in ATS
itself, consisting of more than 100K lines of code, and ATS2
(ATS/Postiats), the successor of ATS is being actively developed and should
be ready for release in the near feature.

\end{itemize}
\end{slide}
%%%%%%
%%
%% Slide 3.2
%%
%%%%%%
\begin{slide}{\em ATS (2)}
\begin{itemize}
\item
The core of ATS is a call-by-value functional programming language,
which can also support lazy evaluation.
\item
ATS supports a programming paradigm referred to as {\em programming with
theorem-proving (PwTP)}, allowing code for run-time computation and
code for proof construction to be written in a syntactically intertwined
manner.
\item
Imperative programming in ATS makes essential use of PwTP.
\item
ATS and C share the same native data representation, and
programs in ATS are compiled directly into C code. To some extent, ATS
can be thought of as a front end for C.
\item
For more details, please visit:
\begin{center}
\texttt{\bf http://www.ats-lang.org}
\end{center}
\end{itemize}
\end{slide}
%%%%%%
%%
%% Slide 4
%%
%%%%%%
\begin{slide}{\em The rest of the talk}
\begin{itemize}
\item Programming with Theorem-Proving (PwTP)
\item Views as Linear Props
\item Viewtypes as Linear Types
\end{itemize}
\end{slide}
%%%%%%
%%
%% Slide 5.1
%%
%%%%%%
\begin{slide}{\em Programming with Theorem-Proving}

Programming with Theorem-Proving (PwTP) is considered the signatory
feature of ATS. It advocates a programming paradigm in which code for
(run-time) computation and code for proof construction can be written in a
syntactically intertwined manner.

\begin{itemize}
\item
Props are introduced to classify proofs.
\item
A proof is required to be pure and total, and it is to be erased before
program execution. In particular, we do not extract programs out of
proofs. Consequently, we can and do construct classical proofs.
\end{itemize}
\end{slide}
%%%%%%
%%
%% Slide 5.2.1
%%
%%%%%%
\begin{slide}{\em Ind. Def. of Fibonacci Numbers}
$$\begin{array}{rcl}
fib(0) & = & 0 \\
fib(1) & = & 1 \\
fib(n) & = & fib(n-2)+fib(n-1)~~\mbox{if $n \geq 2$} \\
\end{array}$$
\end{slide}
%%%%%%
\begin{slide}{\em Implementation of $fib$ in C}
{\blue\begin{verbatim}
int fib (int n) {
  int tmp;
  int r0 = 0, r1 = 1;
  while (n > 0) { tmp = r1; r1 = r0+r1; r0 = tmp; }
  return r0 ;
} // end of [fib]
\end{verbatim}}
\end{slide}
%%%%%%
%%
%% Slide 5.2.3
%%
%%%%%%
\begin{slide}{\em A Dataprop Declaration in ATS}
{\blue\begin{verbatim}
dataprop FIB (int, int) =
  | FIB0 (0, 0) of () // fib(0) = 0
  | FIB1 (1, 1) of () // fib(1) = 1
  | {n:nat}{r0,r1:nat} // fib(n+2) = fib(n)+fib(n+1)
    FIB2 (n+2, r0+r1) of (FIB (n, r0), FIB (n+1, r1))
\end{verbatim}
}~\\~\\
The following function {\it fib} computes Fibonacci numbers:\\~\\
{\blue\begin{verbatim}
fun fib {n:nat}
  (n: int (n)): [r:nat] (FIB (n, r) | int (r))
// end of [fib]
\end{verbatim}
}
\vfill
\end{slide}
%%%%%%
%%
%% Slide 5.3
%%
%%%%%%
\begin{slide}{\em A Direct Implementation of {\it fib}}
{\blue\begin{verbatim}
implement fib (n) =
  case+ n of
  | 0 => (FIB0 () |  0)
  | 1 => (FIB1 () |  1)
  | _ =>> let
      val (pf0 | r0) = fib (n-2)
      val (pf1 | r1) = fib (n-1)
    in
      (FIB2 (pf0, pf1) | r0 + r1)
    end // end of [_]
// end of [fib]
\end{verbatim}
}
\end{slide}
%%%%%%
%%
%% Slide 5.4
%%
%%%%%%
\begin{slide}{\em A Tail-Recursive Implementation of {\it fib}}
{
\fontsize{10}{10}
\blue\begin{verbatim}
implement
fib {n} (n) = let
  fun loop
    {i:nat | i < n} {r0,r1:nat} (
    pf0: FIB (i, r0), pf1: FIB (i+1, r1)
  | ni: int (n-i), r0: int r0, r1: int r1
  ) : [r:nat] (FIB (n, r) | int r) =
    if ni > 1 then
      loop {i+1} (pf1, FIB2 (pf0, pf1) | ni-1, r1, r0+r1)
    else (pf1 | r1)
  // end of [loop]
in
  case+ n of
  | 0 => (FIB0 () |  0)
  | 1 => (FIB1 () |  1)
  | _ =>> loop {0} (FIB0, FIB1 | n, 0, 1)
end // end of [fib]
\end{verbatim}
}
\end{slide}
%%%%%%
%%
%% Slide ??
%%
%%%%%%
\begin{slide}{\em Views}
A view is a linear prop.
\begin{itemize}
\item
Given a type $T$ and an address $L$, $T@L$ is a primitive view
meaning that a value of the type $T$ is stored at the location $L$.
\item
Given two views $V_1$ and $V_2$, we use $V_1\otimes V_2$ for a
view that joins $V_1$ and $V_2$ together.
\item
We also provide a means for forming recursive views. 
\end{itemize}
\end{slide}
%%%%%%
%%
%% Slide ??
%%
%%%%%%
%%%%%%
%%
%% Slide 18
%%
%%%%%%
\def\tptr{\mbox{\bf ptr}}
\def\fptrget{{\it ptrget}}
\def\fptrset{{\it ptrset}}
\begin{slide}{\em Some Built-in Functions}
{\blue\begin{verbatim}
fun{a:type}
ptrget {l:addr}
  (pf: a@l | p: ptr l): (a@l | a)
// end of [ptrget]

fun{a:type}
ptrset {l:addr}
  (pf: (a?)@l | p: ptr l, x: a): (a@l | void)
// end of [ptrset]
\end{verbatim}
}
\[\begin{array}{rcl}
\fptrget &~:~& \forall a:\stype.\forall l:\saddr.~~(a@l \mid \tptr(l)) \timp (a@l \mid a) \\
\fptrset &~:~& \forall a:\stype.\forall l:\saddr.~~(a?@l \mid \tptr(l), a)\timp (a@l \mid \tvoid) \\
\end{array}\]
\end{slide}
%%%%%%
%%
%% Slide ??
%%
%%%%%%
\begin{slide}{\em Example: {\it Swap}}
{\blue\begin{verbatim}
fun{a:type}
swap {l1,l2:addr} (
  pf1: a@l1, pf2: a@l2
| p1: ptr l1, p2: ptr l2
) : (a@l1, a@l2 | void) = let
  val (pf1 | x1) = ptrget<a> (pf1 | p1)
  val (pf2 | x2) = ptrget<a> (pf2 | p2)
  val (pf1 | ()) = ptrset<a> (pf1 | p1, x2)
  val (pf2 | ()) = ptrset<a> (pf2 | p2, x1)
in
  (pf1, pf2 | ())
end // end of [swap]
\end{verbatim}
}
\end{slide}
%%%%%%
%%
%% Slide ??
%%
%%%%%%
\begin{slide}{\em Syntactical Convenience}
{\blue\begin{verbatim}
fun{a:type}
ptrget {l:addr} (pf: !a@l | p: ptr l): a

fun{a:type}
ptrset {l:addr}
  (pf: !(a?)@l >> a@l | p: ptr l, x: a): void
// end of [ptrset]
\end{verbatim}
}
\end{slide}
%%%%%%
%%
%% Slide ??
%%
%%%%%%
\begin{slide}{\em Another Implementation of {\it Swap}}
{\blue\begin{verbatim}
fun{a:type}
swap {l1,l2:addr} (
  pf1: !a@l1, pf2: !a@l2
| p1: ptr l1, p2: ptr l2
) : void = let
  val tmp = !p1 in !p1 := !p2; !p2 := tmp
end // end of [swap]
\end{verbatim}
}
\end{slide}
%%%%%%
%%
%% Slide ??
%%
%%%%%%
\def\VT{{\it VT}}
\def\sviewtype{{\it viewtype}}
\begin{slide}{\em Viewtypes}
\begin{itemize}
\item
A linear type in ATS is given the name viewtype, which is chosen to
indicate that a linear type consists of two parts: one part for views and
the other for types. We use $\sviewtype$ as the sort for viewtypes.
\item Given a view V and a type T, the
tuple (V | T) is a viewtype, where the bar symbol (|) is a separator (just
like a comma) to separate views from types.
\item
What seems a bit surprising is the opposite: For each viewtype VT, we may
assume the existence of a view V and a type T such that VT is equivalent to
(V | T). Formally, this T can be referred as VT?! in ATS. This somewhat
unexpected interpretation of linear types stresses that the linearity of a
viewtype comes entirely from the view part residing within it.
\end{itemize}
\vfill
\end{slide}
%%%%%%
%%
%% Slide ??
%%
%%%%%%
\begin{slide}{\em Abstract Viewtypes (1)}
{\blue\begin{verbatim}
absviewtype queue (a:viewtype, n:int)

fun{a:viewtype} queue_new (): queue (a, 0)
fun{a:viewtype} queue_free (obj: queue (a, 0)): void

fun{a:viewtype}
queue_enque {n:nat} (
  obj: !queue (a, n) >> queue (a, n+1), x: a
) : void // end of [queue_enque]

fun{a:viewtype}
queue_deque {n:pos}
  (obj: !queue (a, n) >> queue (a, n-1)): a
\end{verbatim}
}
\end{slide}
%%%%%%
%%
%% Slide ??
%%
%%%%%%
\begin{slide}{\em Abstract Viewtypes (2)}
{\blue\begin{verbatim}
absviewtype mylock (v:view)

fun mylock_create
  {v:view} (pf: v | (*none*)): mylock(v)

fun mylock_destroy
  {v:view} (lock: mylock v): (v | void)

fun mylock_acquire
  {v:view} (lock: !mylock v): (v | void)

fun mylock_release
  {v:view} (pf: v | lock: !mylock v): void
\end{verbatim}
}
\end{slide}
%%%%%%
%%
%% Slide ??
%%
%%%%%%
\begin{slide}{\em Future Directions}
\begin{itemize}
\item Applying ATS to systems programming. We need more case studies.
\item
Reducing the amount of ``administrative code'' needed for proof
manipulation. We want to investigate more effective approaches to proof
management in order to better support PwTP.
\item
Identifying the potential of PwTP in areas other than systems programming
(e.g., security, privacy).
\end{itemize}
\end{slide}
%%%%%%
%%
%% Slide ??
%%
%%%%%%
\begin{slide}{\em The End}
\vspace{72pt}
\begin{center}
{\huge\bf Thank you!}
\end{center}
\end{slide}
\end{document}
%%%%%% end of [ATSforC.tex] %%%%%%
