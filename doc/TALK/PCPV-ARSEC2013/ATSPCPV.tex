\documentclass[pdf]{prosper}
\usepackage{amssymb}

\renewcommand{\slidetopmargin}{25mm}
\renewcommand{\slidebottommargin}{12mm}
\renewcommand{\slideleftmargin}{0mm}
\renewcommand{\sliderightmargin}{32mm}

%%%%%%
%%
%% Title:
%% A Programmer-Centric Approach to Program Verification
%% Author:
%% Zhiqiang (Alex) Ren and Hongwei Xi, Boston University
%%
%% Date: the 9th of June, 2013
%%
%% Automated Reasoning in Security and Software Verification
%% ARSEC 2013, A Workshop with CADE24
%% Lake Placid, New York, 9 June 2013

%%
\title{%
\huge\bf%
A Programmer-Centric Approach to Program Verification%
}
\author{~\\~\\
{\large Zhiqiang (Alex) Ren and \underline{Hongwei Xi}} \\~\\
{\large Boston University} \\~\\~\\~\\
\institution{Work partly funded by NSF}}
\slideCaption{\footnotesize{\em Views and Viewtypes in ATS}}
%%%%%%
%%
%% Abstract
%%
%% ATS is a statically typed general-purpose programming language.  The
%% signatory feature of ATS is a programming paradigm named
%% programming-with-theorem-proving in which code for (run-time) computation
%% and code for proof construction can be written in a syntactically
%% intertwined manner.  In ATS, there is direct support for both dependent
%% types and linear types. While the dependent types in ATS, which are of a
%% restricted form originated from the development of Dependent ML (DML), are
%% well-studied, the linear types in ATS are much less well-known.  In this
%% talk, I will give an introduction to a notion of linear types referred to
%% as viewtypes in ATS, which combine views (i.e., linear types for proofs)
%% and types (for run-time values). In addition, I will present several
%% concrete examples to illustrate certain practical uses of views and
%% viewtypes.
%%
%%%%%%
\begin{document}
%%%%%%
%%
%% Slide 1
%%
%%%%%%
\maketitle
%%%%%%
%%
%% Slide 2
%%
%%%%%%
\begin{slide}{\em ATS(1)}
\begin{itemize}
\item
ATS is a statically typed programming language that unifies implementation
with formal specification.
\item
The type system of ATS is rooted in the framework {\em Applied Type System},
which gives the language its name.
\item
Both dependent types (DML-style) and linear types are available in ATS.
\item
The current compiler of ATS (ATS/Anairiats) is written in ATS itself,
consisting of approximately 100K lines of code.
\item
ATS2, the next generation of ATS, is under active development and its
compiler (ATS/Postiats) is planned to be released during the summer of
2013.
\end{itemize}
\end{slide}
%%%%%%
%%
%% Slide 3
%%
%%%%%%
\begin{slide}{\em ATS(2)}
\begin{itemize}
\item
The core of ATS is a call-by-value functional programming language, which
can also support lazy evaluation.
\item
ATS supports a programming paradigm referred to as {\em programming with
theorem-proving (PwTP)}, allowing code for run-time computation and
code for proof construction to be written in a syntactically intertwined
manner.
\item
Imperative programming in ATS makes essential use of PwTP.
\item
ATS and C share the same native data representation, and
programs in ATS are compiled directly into C code. To some extent, ATS
can be thought of as a front-end to C.
\item
For more details, please visit:
\begin{center}
\texttt{http://www.ats-lang.org}
\end{center}
\end{itemize}
\end{slide}
%%%%%%
%%
%% Slide 4
%%
%%%%%%
\def\ATS{\mbox{$\cal A\kern-1.5pt T\kern-3pt S$}}
\begin{slide}{\em Applied Type System ($\ATS$)}
\begin{itemize}
\item
$\ATS$ is a framework developed to facilitate designing and
formalizing (advanced) type systems in support of practical
programming.
\item
The name {\em applied type system} refers to a type system formed in
$\ATS$, which consists of two components:
\begin{itemize}
\item
static component (statics), where types are formed and reasoned about.
\item
dynamic component (dynamics), where programs are constructed and evaluated.
\end{itemize}
\item
The key salient feature of $\ATS$ is that statics is completely separate
from dynamics. In particular, types cannot depend on programs.
\end{itemize}
\end{slide}
%%%%%%
%%
%% Slide 5
%%
%%%%%%
\begin{slide}{\em Examples and Non-Examples of ATS's}
\begin{itemize}

\item Examples:
\begin{itemize}
\item
The simply-typed $\lambda$-calculus
\item Dependent ML (DML)
\item
The 2nd-order polymorphic $\lambda$-calculus (System $F$)
\item
The higher-order polymorphic $\lambda$-calculus (System $F_{\omega}$)
\item
System $F_c$ (System $F$ extended with generalized datatypes)
\end{itemize}

\item Non-Examples:
\begin{itemize}
\item The dependent $\lambda$-calculus ($\lambda P$)
\item The calculus of constructions ($\lambda C$) and its variants
\end{itemize}

\end{itemize}
\end{slide}
%%%%%%
%%
%% Slide 6
%%
%%%%%%
\def\sc{{\it sc}}
\def\basesort{b}
\def\saddr{{\it addr}}
\def\sbool{{\it bool}}
\def\simp{\rightarrow}
\def\Simp{\Rightarrow}
\def\sint{{\it int}}
\def\snat{{\it nat}}
\def\sprop{{\it prop}}
\def\stype{{\it type}}
%% \begin{slide}{{\em Syntax for Statics}}
%% \begin{itemize}
%% \item
%% The statics is a simply typed language and a type in the statics is
%% referred to as a {\em sort}. We write $b$ for a base sort and assume the
%% existence of two special base sorts $\stype$ and $\sbool$.
%% \[\begin{array}{lrcl}
%% \mbox{sorts} & \sigma & ::= & \basesort\mid\sigma_1\simp\sigma_2 \\
%% \mbox{c-sorts} & \sigma_c & ::= & (\sigma_1,\ldots.\sigma_n)\Simp\sigma \\
%% \mbox{sta.cd  terms} & s & ::= & a \mid \sc(s_1,\ldots,s_n) \mid \lambda a:\sigma.s \mid s_1(s_2) \\
%% \mbox{sta. var. ctx.} & \Sigma & ::= & \emptyset \mid \Sigma, a:\sigma \\
%% \end{array}\]
%% \item
%% In practice, we also have base sorts $\sint$ and $\saddr$ for integers and
%% addresses (or locations), respectively.  Let us use $B$, $I$, $L$ and $T$
%% for static terms of sorts $\sbool$, $\sint$, $\saddr$ and $\stype$,
%% respectively.
%% \end{itemize}
%% \end{slide}
%%%%%%
%%
%% Slide 7
%%
%%%%%%
\def\Band{\land}
\def\Bimp{\supset}
\def\ttrue{{\it true}}
\def\ffalse{{\it false}}
\def\timp{\rightarrow}
\def\Timp{\Rightarrow}
\def\tbool{\mbox{\bf bool}}
\def\tint{\mbox{\bf int}}
\def\tleq{\leq}
\def\tpjg{\vdash}
\def\tunit{{\mathbf 1}}
%% \begin{slide}{\em Some Static Constants}
%% \[\begin{array}{rcll}
%% \\[-24pt]
%% \tunit & ~:~ & ()\Simp \stype \\[4pt]
%% \ttrue & ~:~ & ()\Simp \sbool \\[4pt]
%% \ffalse & ~:~ & ()\Simp \sbool \\[4pt]

%% \timp & ~:~ & (\stype,\stype) \Simp \stype \\[4pt]
%% \Bimp & ~:~ & (\sbool,\stype) \Simp \stype \\[4pt]

%% \Band & ~:~ & (\sbool,\stype) \Simp \stype \\[4pt]
%% \tleq & ~:~ & (\stype,\stype) \Simp \sbool & \mbox{(impredicative formulation)}\\[4pt]
%% \end{array}\]
%% Also, for each sort $\sigma$, we assume that the two static constructors
%% $\forall_\sigma$ and $\exists_\sigma$ are assigned the sc-sort
%% $(\sigma\timp\stype)\Simp\stype$.
%% \vfill
%% \end{slide}
%%%%%%
%%
%% Slide 8
%%
%%%%%%
\def\vB{\vec{B}}
\def\temd{\models}
\begin{slide}{\em Constraint Relation}
A constraint relation is of the following form:
\[\begin{array}{c}
\Sigma;\vB\temd B
\end{array}\]
where $\vB$ stands for a sequence of static boolean terms
(often referred to as assumptions). This relation holds if
one of the assumptions in $\vB$ is false or the conclusion $B$
is true.\\~\\
\end{slide}
%%%%%%
%%
%% Slide 9
%%
%%%%%%
\begin{slide}{\em A Sample Constraint}
\def\sz{{\it sz}}
The following constraint is generated when an
implementation of binary search on arrays is type-checked:
\[\begin{array}{rcl}
\Sigma;\vB & \temd & l+(h-l)/2+1\leq \sz \\
\end{array}\]
where
\[\begin{array}{rcl}
\Sigma & = & h:\sint, l:\sint, \sz:\sint \\
\vB & = & l\geq 0, \sz\geq 0, 0\leq h+1, h+1\leq \sz, 0\leq l, l\leq \sz, h\geq l \\
\end{array}\]
We may employ linear integer programming to solve such a constraint.\\~\\
\end{slide}
%%%%%%
%%
%% Slide 10
%%
%%%%%%
\begin{slide}{\em Some (Unfamiliar) Forms of Types}
\begin{itemize}
\item Asserting type: $B\Band T$
\item Guarded type: $B\Bimp T$
\end{itemize}
Here is an example involving both guarded and asserting types:
\[\begin{array}{l}
\forall a:\sint.~a \geq 0\Bimp (\tint (a) \timp \exists a':\sint.~(a'<0)\Band \tint(a')) \\
\end{array}\]
This type can be assigned to a function from nonnegative integers to
negative integers. As a probably more interesting example, the usual
run-time assertion function can be given the following type:
\[\begin{array}{l}
\forall a:\sbool.~\tbool(a)\timp(a=\ttrue)\Band\tunit \\
\end{array}\]
\vfill
\end{slide}
%%%%%%
%%
%% Slide 11
%%
%%%%%%
\def\dc{{\it dc}}
\def\dcc{{\it dcc}}
\def\dcf{{\it dcf}}
\def\iforall#1{\forall^{+}(#1)}
\def\eforall#1{\forall^{-}(#1)}
\def\iguard#1{\supset^{+}\kern-2pt(#1)}
\def\eguard#1{\supset^{-}\kern-2pt(#1)}
\def\lam#1#2{\mbox{\bf lam}\;#1.#2}
\def\app#1#2{\mbox{\bf app}(#1,#2)}
\def\letin#1#2{\mbox{\bf let}\;#1\;{\bf in}\;#2}
%% \begin{slide}{\em Syntax for Dynamics}
%% \[\begin{array}{lrcl}
%% \mbox{dyn. terms} & d & ::= & x \mid \dc(d_1,\ldots,d_n) \mid \\
%% & & & \lam{x}{d} \mid \app{d_1}{d_2} \mid \\
%% & & & \iguard{v} \mid \;\eguard{d} \mid \\
%% & & & \iforall{v} \mid \eforall{d} \mid \\
%% & & & \Band(d) \mid \letin{\Band(x)=d_1}{d_2} \mid \\
%% & & & \exists(d) \mid \letin{\exists(x)=d_1}{d_2} \\
%% \mbox{values} & v & ::= & x \mid \dcc(v_1,\ldots,v_n) \mid \lam{x}{d} \mid \\
%% & & & \iguard{v} \mid \iforall{v} \mid \Band(v) \mid \exists(v) \\
%% \mbox{dyn. var. ctx.} & \Delta & ::= & \emptyset \mid \Delta, x:s \\
%% \end{array}\]
%% \end{slide}
%%%%%%
%%
%% Slide 12
%%
%%%%%%
\begin{slide}{\em Typing Judgment}
A typing judgment is of the following form:
\[\begin{array}{c}
\Sigma;\vB;\Delta\tpjg d:T
\end{array}\]
where
\[\begin{array}{rcl}
\Sigma & : & \mbox{static variable context} \\
\vB & : & \mbox{assumption set} \\
\Delta & : & \mbox{dynamic variable context} \\
d & : & \mbox{dynamic term} \\
T & : & \mbox{type} \\
\end{array}\]
\end{slide}
%%%%%%
%%
%% Slide 13
%%
%%%%%%
\begin{slide}{\em A Datatype Declaration in ATS}
{\blue\begin{verbatim}
datatype list (a:type, int) =
  | nil (a, 0) of ()
  | {n:int | n >= 0} cons (a, n+1) of (a, list (a, n))
\end{verbatim}
}~\\~\\
The concrete syntax means the following:
\[\begin{array}{rcl}
nil & : & \forall a:\stype.~~() \Timp list (a, 0) \\
cons & : & \forall a:\stype.\forall n:\sint. \\
     &   & \kern12pt n\geq 0 \Bimp ((a, list (a, n)) \Timp list (a, n+1)) \\
\end{array}\]
\vfill
\end{slide}
%%%%%%
%%
%% Slide 14
%%
%%%%%%
\begin{slide}{\em Programming with Theorem-Proving}
\begin{itemize}
\item
We introduce a new sort $\sprop$ into the statics and use $P$ for static
terms of sort $\sprop$, which are often referred to as props.
\item
A prop is like a type, which is intended to be assigned to special dynamic
terms that we refer to as proof terms.
\item
A proof term is required to be pure and total, and it is to be erased
before program execution. In particular, we do not extract programs out of
proofs. Consequently, we can and do construct classical proofs.
\end{itemize}
\end{slide}
%%%%%%
%%
%% Slide 15
%%
%%%%%%
\begin{slide}{\em A Dataprop Declaration in ATS}
{\blue\begin{verbatim}
datatype FIB (int, int) =
  | FIB0 (0, 0) of () // fib(0) = 0
  | FIB1 (1, 1) of () // fib(1) = 1
  | {n:nat}{r0,r1:nat} // fib(n+2) = fib(n)+fib(n+1)
    FIB2 (n+2, r0+r1) of (FIB (n, r0), FIB (n+1, r1))
\end{verbatim}
}~\\~\\
The following function {\it fib} computes Fibonacci numbers:\\~\\
{\blue\begin{verbatim}
fun fib {n:nat}
  (n: int (n)): [r:nat] (FIB (n, r) | int (r))
// end of [fib]
\end{verbatim}
}
\vfill
\end{slide}
%%%%%%
%%
%% Slide 16
%%
%%%%%%
\begin{slide}{\em A Direct Implementation of {\it fib}}
{\blue\begin{verbatim}
implement fib (n) =
  case+ n of
  | 0 => (FIB0 () |  0)
  | 1 => (FIB1 () |  1)
  | _ =>> let
      val (pf0 | r0) = fib (n-2)
      val (pf1 | r1) = fib (n-1)
    in
      (FIB2 (pf0, pf1) | r0 + r1)
    end // end of [_]
// end of [fib]
\end{verbatim}
}
\end{slide}
%%%%%%
%%
%% Slide 16-2
%%
%%%%%%
\begin{slide}{\em A Linear-Time Implementation of {\it fib}}
{
\blue\begin{verbatim}
fun fibats (n: int): int = let
  fun loop (r0: int, r1: int, ni: int): int =
    if ni > 0 then loop (r1, r0+r1, ni-1) else r0
  // end of [loop]
in
  loop (0, 1, n)
end // end of [fibats]
\end{verbatim}
}
\end{slide}
%%%%%%
%%
%% Slide 16-2
%%
%%%%%%
\begin{slide}{\em A Linear-Time Implementation of {\it fib}}
{
\fontsize{11pt}{12pt}
\blue\begin{verbatim}
fun fibats {n:nat} (n: int n)
  : [r:int] (FIB (n, r) | int r) = let
  fun loop {n,i:nat | i <= n} {r0,r1:int} .<n-i>.
  (
    pf0: FIB (i, r0)
  , pf1: FIB (i+1, r1)
  | r0: int (r0), r1: int (r1), ni: int(n-i)
  ) : [r:int] (FIB (n, r) | int (r)) =
    if ni > 0 then
      loop {n,i+1} (pf1, FIB2 (pf0, pf1) | r1, r0+r1, ni-1)
    else (pf0 | r0) // end of [if]
in
  loop (FIB0(), FIB1() | 0, 1, n)
end // end of [fibats]
\end{verbatim}
}
\end{slide}
%%%%%%
%%
%% Slide 17-1
%%
%%%%%%
\begin{slide}{\em Program-First Program Verification}
\end{slide}
%%%%%%
%%
%% Slide 17-2
%%
%%%%%%
\begin{slide}{\em Programmer-Centric Program Verification}
\begin{itemize}
\item Observation: (informal) proofs are robust but programs are fragile
\item Observation: theorems are almost always correct but proofs almost always contain flaws.
\end{itemize}
\end{slide}
%%%%%%
%%
%% Slide 18-1
%%
%%%%%%
\begin{slide}{\em Datatype for Generic Functional Lists}
{\blue
\fontsize{11}{12}
\begin{verbatim}
datatype
gflist (
  a:type, ilist(*index*)
) =
  | gflist_nil (a, nil) of ()
  | {x:int} {xs:ilist}
    gflist_cons (a, cons (x, xs)) of (E (a, x), gflist (a, xs))
// end of [gflist]
\end{verbatim}
}
\end{slide}
%%%%%%
%%
%% Slide 18-2
%%
%%%%%%
\begin{slide}{\em Interface for List-Sorting}
{\blue
\fontsize{11}{12}
\begin{verbatim}
fun{a:type}
sort{xs:ilist}
(
  xs: glist (a, xs)
, compare: (a, a) -> int
) : [ys:ilist]
(
  SORT (xs, ys) | glist (a, ys)
)
// end of [sort]
\end{verbatim}
}
\end{slide}
%%%%%%
%%
%% Slide 18-3
%%
%%%%%%
\begin{slide}{\em Verified Insertion Sort}
{
\begin{itemize}
\item Let's take a look at an implementation of list-insertion sort.
\end{itemize}
}
\end{slide}
%%%%%%
%%
%% Slide 35
%%
%%%%%%
\begin{slide}{\em Conclusion}
\begin{itemize}
\item
ATS is a statically type programming language that unifies implementation
with a form of formal specification.
\item
The signatory feature of ATS is a programming paradigm referred to as
programming-with-theorem-proving (PwTP) in which code for (run-time)
computation and code for proof construction can be written in a
syntactically intertwined manner.
\item
We advocate a programmer-centric approach to program verification
in ATS that puts emphasis on requesting the programmer to explain in
a literate fashion why his or her code works.
\end{itemize}
\end{slide}
%%%%%%
%%
%% Slide 36
%%
%%%%%%
\begin{slide}{\em Future Direction}
\begin{itemize}
\item Applying ATS to systems programming. We need more case studies.
\item
Reducing the amount of ``administrative code'' needed for proof
manipulation. We want to investigate more effective approaches to proof
management in order to better support PwTP.
\item
Identifying the potential of PwTP in areas other than systems programming
(e.g., security).
\end{itemize}
\end{slide}
%%%%%%
%%
%% Slide 37
%%
%%%%%%
\begin{slide}{\em The End}
\vspace{72pt}
\begin{center}
{\huge\bf Thank you!}
\end{center}
\end{slide}
\end{document}
%%%%%% end of [ViewsViewtypes.tex] %%%%%%
